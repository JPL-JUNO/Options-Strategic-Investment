\documentclass{article}
\usepackage{ctex}
\usepackage{amssymb, amsmath}
\usepackage[margin=1in]{geometry}
\usepackage{graphicx}
\usepackage{enumitem}
\setlist[enumerate]{topsep=.5ex, partopsep=0ex, parsep=.5ex, itemsep=0ex}
\usepackage{mathptmx}
\usepackage{tcolorbox}
\tcbset{colframe=red!50!white}
\begin{document}
在许多方面,看跌期权以及同它相关的策略都几乎完全是那些以看涨期权为基础的策略的反面。不过,看跌期权完全是看涨期权的反面这种说法是不正确的。
\section{看跌期权策略}
就其最简单的形式而言,在直接买入看跌期权时,交易者是希望股票下跌,从而他的看跌期权会变得更值钱。如果股票下跌到看跌期权的行权价之下足够远的地方,看跌期权的持有者就会盈利。这手看跌期权的持有者可以在公开的市场买入股票,然后将他的看跌期权行权,从而按照比股票价格更高的行权价卖出股票,取得盈利。
\end{document}