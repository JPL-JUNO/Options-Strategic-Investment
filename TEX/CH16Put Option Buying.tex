\chapter{买入看跌期权}
\section{选择要买的看跌期权}
在确定究竟买入哪个看涨期权时所采用的许多分析方法,也可以用在看跌期权的买入选择上。首先,在把买入看跌期权用作投机策略的时候,交易者投入到这个策略的资金不应当高于其 15\% 的风险资本。

相对于实值看跌期权,虚值看跌期权的潜在收益更高,潜在风险也更大。如果标的股票大幅下跌,买入 1 手更便宜的虚值看跌期权所提供的百分比收益就更大。不过,如果标的股票的价格下跌幅度不大,那么实值看跌期权就常常是更好的选择。事实上,看跌期权在变为实值后时间价值往往失去得更为迅速,买入实值看跌期权甚至有更大的好处。

\begin{tcolorbox}
    XYZ 的价格是 49,有以下的价格存在。
    \begin{itemize}
        \item XYZ 普通股股票:49
        \item XYZ 7 月 45 看跌期权:1
        \item XYZ 7 月 50 看跌期权:3
    \end{itemize}

    如果标的股票在到期时下跌到 40,7 月 45 看跌期权就会价值 5 点,这是 400\% 的盈利。7 月 50 看跌期权就会价值 10 点,相对于最初 3 点的购买价格,这是 233\% 的盈利。因此,在大幅度的下跌运动中,买入虚值看跌期权提供了更高的潜在盈利。不过,如果标的股票只有中等幅度的下跌,例如,跌到 45,买入 7 月 45 看跌期权就会亏损掉全部的投资,因为这个看跌期权在到期时会一文不值。如果 XYZ 在到期时的价格为 45,7 月 50 看跌期权的买家就会有 2 点的盈利。

    \textbf{前面的分析是以将看跌期权一直持有至到期日为基础的。对期权的买家来说,这通常是一种错误的分析,因为买家常常在到期之前就将他买入的期权平仓。}在考虑到期之前看跌期权会发生什么时,记住这样的事实是有用的:实值看跌期权倾向于更迅速地失去它的时间价值。在上面的示例里,7 月 45 看跌期权完全由时间价值组成。如果标的股票跌到 45 之下,\textbf{这个看跌期权的价格不会像变为实值的看涨期权的价格涨得那么快。}
\end{tcolorbox}

如果 XYZ 下跌了 5 点,价格到了 44,那么,这显然是一个对看跌期权买家有利的运动,结果他发现 7 月 45 看跌期权的价值只增加了 2 点或 2.50 点。这多少令人失望,因为如果是看涨期权,在股票有了对他有利的 5 点的运动之后,交易者会期望有更多的增值。因此,\textcolor{red}{在以投机的目的买入看跌期权的时候,除非预期标的股票价格会有很大幅度的下降,否则一般最好将注意力集中在实值看跌期权上。}
\section{止损行动}
在期权买入的策略中常常出现的情况是,期权的持有者面临未兑现的亏损。在这种情况里,看跌期权持有者也有若干可供选择的行动。他首要的和最简单的做法是将这手看跌期权卖掉,承担亏损。虽然在一定的情况里应当这样做,特别是当标的股票似乎已经显然要朝看多的方向继续运动下去的时候,不过,这并不总是最明智的做法。有亏损的看跌期权持有者也可以考虑采用“向上挪仓”以构造一手看空的价差,或者建立一手跨期价差。这两种方法都可以帮助他挽回部分亏损。
\subsection{“向上挪仓”策略}
\begin{tcolorbox}
    一个投资者在标的股票为 45 时用 3 点买入了一手 XYZ 10 月 45 看跌期权。但是,股票后来上涨到了 48,最初用 3 点买入的看跌期权现在只值 1.50 点。顺便说一下,一手看跌期权即使在过了一段时间并且股票在上涨的情况下仍然保留了这么多的价值,这并不奇怪,因为虚值看跌期权在保存时间价值方面往往做得相当好。当 XYZ 为 48 时,10 月 50 看跌期权的售价有可能是 3 点。看跌期权持有者可以通过卖出 2 手他现在持有的看跌期权(10 月 45),同时买入 1 手10 月 50 看跌期权,来构造出一个用来挽回部分亏损的头寸。
\end{tcolorbox}

建立这个价差的效果是,投资者的风险一点也没有增加,但是他的头寸的盈亏平衡点被提高了。也就是说,只要 XYZ 小幅度下跌,他就有可能实现盈亏平衡。如果没有建立这个价差,这个看跌期权持有者需要 XYZ 在到期时跌回到 42 才能实现盈亏平衡,因为他开始时为 10 月 45 看跌期权支付了 3 点。他的初始风险是 300 美元。如果 XYZ 价格继续上涨,价差中的看跌期权就无价值到期,净亏损仍然只是 300 美元加额外的手续费。应当承认,价差的手续费会略为增加亏损,不过,同这个头寸的支出(300 美元)相比,它的数量很小。另一方面,如果股票在到期日时小幅下跌到 47,那么,这个价差就会实现盈亏平衡。在到期时,如果 XYZ 是 47,实值的 10 月 50 看跌期权就价值 3 点,虚值的 10 月 45 看跌期权就会无价值到期。因此,当 XYZ 的价格在期权到期时是 47,不计手续费,这个投资者就会收回 300 美元。他的盈亏平衡点从 42 提到了 47,这显著地改善了他恢复盈亏平衡的机会。

总的来说,\textcolor{red}{有未兑现亏损的看跌期权持有者可以通过卖出两倍数量于他目前持有的看跌期权和买入次高行权价的看跌期权来构造一手价差。}\textbf{不过,只有当这个交易可以用很小的支出或者最好没有支出来完成时才应当采取这样的做法。}这个价差提供了高得多的实现盈亏平衡的机会,同时也减少了出现头寸最大亏损的可能性。不过,如果标的股票价格后来大幅下跌,引进这些止损的方法会减小这个头寸的最大潜在盈利。
\subsection{跨期价差策略}
有未兑现亏损的看跌期权持有者有时也可以使用另一种策略。如果他所持的是中期或长期的看跌期权,他可以就目前持有的这手看跌期权卖出近期的看跌期权,从而构造一手跨期价差。
\begin{tcolorbox}
    在股票价格为 45 时,投资者用 3 点买入了 1 手 10 月 45 看跌期权。股票上涨到 48,朝着对看跌期权买家不利的方向运动,他的看跌期权的价值跌到 1.50。在这个时候,他可以考虑按 1 点卖出近期的 7 月 45 看跌期权。理想的情况是 7 月 45 看跌期权无价值到期,从而他持有的看跌期权的成本就减少 1 点。这时,如果标的股票下跌到 45 之下,他在 7 月到期日之后还可以盈利。
\end{tcolorbox}

这个策略的主要缺点是,如果标的股票在近期的 7 月期权过期之前跌回到 45 或者更低,那么就只有很少或者没有盈利可言,事实上,有相当大的可能会出现亏损。

这种价差没有“向上挪仓”那么具有吸引力。在“向上挪仓”策略里,如果股票在价差建立之后价格下跌,交易者并不会有亏损,尽管他确实限制了他的盈利。即使股票下跌还会导致亏损这个事实使得跨期价差变得不那么受欢迎。