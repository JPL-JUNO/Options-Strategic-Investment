\chapter{看跌期权的基本原理\label{CH15}}
在许多方面,看跌期权以及同它相关的策略都几乎完全是那些以看涨期权为基础的策略的反面。不过,看跌期权完全是看涨期权的反面这种说法是不正确的。
\section{看跌期权策略}
就其最简单的形式而言,在直接买入看跌期权时,交易者是希望股票下跌,从而他的看跌期权会变得更值钱。如果股票下跌到看跌期权的行权价之下足够远的地方,看跌期权的持有者就会盈利。这手看跌期权的持有者可以在公开的市场买入股票,然后将他的看跌期权行权,从而按照比股票价格更高的行权价卖出股票,取得盈利。

\begin{table}[!ht]
    \centering
    \caption{看涨期权同看跌期权的比较}
    \label{tbl:15-1}
    \begin{tabular}{rrrrrrr}
        \hline
        $S_t$ & c     & 看涨期权内在价值 & 看涨期权时间价值                                     & p    & 看跌期权内在价值 & 看跌期权时间价值 \\ \hline
        40    & 0.5   & 0        & 0.5                                          & 9.75 & 10       & -0.25    \\
        43    & 1     & 0        & 1                                            & 7    & 7        & 0        \\
        45    & 2     & 0        & 2                                            & 6    & 5        & 1        \\
        47    & 3     & 0        & 3                                            & 5    & 3        & 2        \\
        50    & 5     & 0        & 5                                            & 4    & 0        & 4        \\
        53    & 7     & 3        & 4                                            & 3    & 0        & 3        \\
        55    & 8     & 5        & 3                                            & 2    & 0        & 2        \\
        57    & 9     & 7        & 2                                            & 1    & 0        & 1        \\
        60    & 10.5  & 10       & 0.5                                          & 0.5  & 0        & 0.5      \\
        70    & 19.75 & 20       & -0.25\footnote{深度实值期权在到期之前有可能会以低于内在价值的价格交易。} & 0.25 & 0        & 0.25     \\ \hline
    \end{tabular}
\end{table}

\autoref{tbl:15-1} 说明了若干基本事实。随着股票下跌,看涨期权的实际价格下跌,而看跌期权的价值上升。反过来,随着股票上涨,看涨期权会增值,而看跌期权会减值。\textbf{看跌期权和看涨期权都是在当股票价格刚好等于行权价时具有最大的时间价值}。不过,当股票价格等于行权价时,看涨期权的售价一般比看跌期权要高。请注意一下 \autoref{tbl:15-1},当 XYZ 等于 50 时,看涨期权价值 5 点,而看跌期权只值 4 点。除了股息很高的股票之外,一般而言都是如此。这种现象同持有股票的成本有关。\autoref{tbl:15-1} 同时也描写了一种一般来说都存在的看跌期权效应:\textbf{相对于实值看涨期权来说,实值看跌期权(股票低于行权价)失去时间价值的速度要更快。}注意一下 \autoref{tbl:15-1} 中价格为 43 的 XYZ,看跌期权在实值 7 点时就失去了它所有的时间价值。而当看涨期权在 7 点实值的时候,也就是 XYZ 为 57 的时候,还有 2 点时间价值。同样地,这个现象也受到标的股票股息支付的影响,不过,总的来说这个现象都是存在的。
\section{股息对看跌期权权利金的效应}
股息越大,看跌期权的价值就越大。就应该是这样,因为股票的除息会让股票价格减少同股息相同的金额。这就是说,股票价格会下跌,因此看跌期权就会变得更有价值。所以,看跌期权的买家就愿意为这个看跌期权付更高的价格,看跌期权的卖家也将要价更高。同场内看涨期权一样,场内看跌期权也不会因为标的股票支付现金股息而进行调整。不过,期权自身的价格会反映出股票的股息支付。
\begin{tcolorbox}
    XYZ 每股的售价是 25 美元,而且在将来的 6 个月里将支付 1 美元的股息。这样的话,无论任何其他同这个标的股票相关的因素是什么状况,一手 6 个月的行权价为 25 的看跌期权就会至少价值 1 美元。在以后的 6 个月里,这个股票价格会因为股息分发而下跌同股息数量相等的幅度,也就是 1 美元,如果其他条件保持不变,这时这个股票的价格就会是 24 美元。当股票价格为 24 美元时,这个看跌期权就会有 1 美元的实值,因此,它的价值就会至少是 1 个点的内在价值。标的股票将要支付的这笔大额股息,会让该股票的看跌期权的价格提前上涨。
\end{tcolorbox}