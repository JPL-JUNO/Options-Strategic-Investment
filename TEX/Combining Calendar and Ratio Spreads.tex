\chapter{跨期价差和比率价差的组合}
\section{比率跨期价差}
比率跨期价差(ratio calendar spread)是\nameref{CH:calendar spreads}和\href{CH:ratio call spreads}{比率价差}的组合。在比率跨期价差中,价差者卖出一定数量的近期看涨期权,同时买入数量较少的中期或远期看涨期权。因为卖出的期权数量要多于买入的期权,其中就有裸期权。在建立一手比率跨期价差时常常会有收入。也就是说,如果标的股票始终没有上涨到行权价之上,策略家仍然会有盈利。不过,因为涉及裸看涨期权,实施这种策略的质押要求会很高。

使用下列的价格:
\begin{tcolorbox}
    XYZ 普通股股票:45,XYZ 4 月 50 看涨期权:1,XYZ 7 月 50 看涨期权:1.50
\end{tcolorbox}

比率跨期价差可以按下面的方式设立起来:按 1.5 买入 1 手 XYZ 7 月 50 看涨期权,按 1 卖出 2 手 XYZ 4 月 50 看涨期权,净投资 0.5 收入。

如果股票在 4 月到期之前停留在 50 之下,那手 7 月 50 看涨期权多头就会是免费拥有的。在这之后,无论标的股票发生什么情况,这手价差都不会亏损。事实上,如果标的股票在近期到期日之后急剧上涨,随着 7 月 50 看涨期权的增值,会积累大笔盈利。当然,这全都取决于这手 4 月 50 看涨期权是否会无价值到期。如果标的股票在 4 月看涨期权到期之前涨过了 50,因为有裸看涨期权,这个比率跨期价差就有大笔亏损的危险,必须采取防御行动。

如果标的股票在 7 月到期之前停留在 50 之下,策略家可以得到 50 美元的收入,减去手续费。但策略在上行方向的结果则无法得到如此明确的结论。如果股票价格在 4 月 50 看涨期权无价值到期之后上涨,那么潜在盈利的唯一限制就是时间。最让人担心的是股票在 4 月到期日之前上涨。如果股票立刻上涨,那么这个价差无疑会有亏损。如果股票慢慢地涨,但还是在 4 月到期之前达到了 50,那么,这个价差就不会有多大的变化。使用同一个示例。假定 XYZ 在 4 月 50 看涨期权的存续期只剩几个星期的时候涨到了 50。这时 4 月 50 看涨期权的售价有可能是 1.50,7 月 50 看涨期权的售价有可能是 3。这个比率价差在这时平仓有可能保持收支相抵,买回 2 手 4 月 50 看涨期权的支出同卖出 1 手 7 月 50 看涨期权的收入相等。于是在整个交易中,他就有了 0.50 的盈利,再减去手续费。最后,交易者可以估计在4 月 50 看涨期权的到期日,他要实现盈亏平衡时的股票价格。假定他认为 XYZ 在 4 月到期时的价格是 53,那 7 月 50 看涨期权的售价就可能是 5.50。因为 4 月 50 看涨期权在这时的售价是 3(它们会处于持平),要把这个比率价差平仓,就会有 1/2 点的支出。买回这 2 手 4 月 50 看涨期权会有 6 点支出,而卖出 7 月 50 看涨期权有 5.50 点收入,净支出就是 0.50 点。当股票在 4 月到期时为 53,整个价差交易就会实现盈亏平衡,因为建立头寸时有 0.50 点收入,而平仓时有 0.50 点支出。不过要减去手续费。因此,这个价差的风险明显地取决于在 4 月到期之前股票上涨到 50 之上的速度有多快。(近月期权上涨到行权价的速度)

\section{选择价差}
建立牛市跨期价差时所使用的相同标准也可以用在这里。选择波动率较大,且股价能在给定的时间内(在近期合约到期之后,但是在买入的期权到期之前)会超过行权价的股票。不要选那些过于虚值,股票基本上没有希望达到它的行权价的看涨期权。始终用收入(包括手续费)来建立这个价差,这就可以保证即使股票没有运动,也可以盈利。不过,如果需要使用很大的比率(达到每买入 1 手期权就要卖出 3 手)才能产生收入的话,你也许应当拒绝这种选择,因为如果股票立刻上涨,潜在的亏损会非常大。

\section{后续行动}
这个策略中防御行动的主要目的,是限制当股票在 4 月到期之前出现上涨时的亏损。策略家应当在出现严重亏损前迅速将这个价差平仓。一条通常有用的规则是,如果股票突破了技术阻力位,或者高于到期时的最终盈亏平衡点,那么就将这手价差平仓。
\subsection{好的概率}
只要坚持进行上面介绍的防御行动,这个策略盈利的概率就相当高。如果股票价格一直没有超过行权价,这个价差可以盈利,因为这个价差是用收入建立起来的。这种情况本身就有很大的可能性,因为股票最初是低于行权价的。此外,如果股票在近期看涨期权到期之后上涨,这个价差就有很大的潜在盈利。虽然出现这种情况的可能性要小得多,由此产生的盈利可以增进这个价差的预期收益率。这个价差唯一会出现亏损的情形是股票价格迅速上涨,如果出现这种情况,策略家就不得不将这个价差平仓以限制亏损。

\section{delta 中性跨期价差}
用价差中看涨期权的 delta 可以构建更为精密的比率。这种价差可以用虚值看涨期权或者实值看涨期权来建立。前者有裸看涨期权,后者有额外的看涨期权多头。

在这两种情况里,就买入的每一手看涨期权而卖出的看涨期权的数目,是通过用看涨期权空头的 delta 除以看涨期权多头的 delta 而决定的。所有的比率价差都是如此,并不只是跨期价差。

假定 XYZ 的交易价是 45,交易者考虑要使用 7 月 50 看涨期权和 4 月 50 看涨期权来建立一手比率跨期价差。此外,假设这里使用的 7 月看涨期权的 delta 是 0.25,4 月合约的 delta 是 0.15。有了这些信息,交易者可以计算出中性的比率是 1.667(0.25/0.15)。这就是说,交易者就每一手买入的看涨期权要卖出 1.667 手看涨期权;换句话说,每买入 3 手就要卖出 5 手。

这种虚值中性跨期价差相当典型。当看涨期权是虚值的时候,交易者往往卖出的看涨期权数量要大于买入的看涨期权数量,以建立一个中性的跨期价差。对实值看涨期权来说,较短期的看涨期权的 delta 要高于较长期的看涨期权。
\subsection{实值跨期价差}
当看涨期权是实值的时候,中性价差看上去明显不同。用一个示例可以帮助说明这个问题。

\begin{tcolorbox}
    XYZ 的交易价是 49,交易者想要使用 7 月 45 和 4 月 45 看涨期权来建立一手中性的跨期价差。这些实值看涨期权的 delta 是:4 月的合约为 0.8,7 月合约为 0.7。
\end{tcolorbox}

这种类型的头寸有可能很有吸引力。首先,它没有上行方向的风险;股价上涨时,实值跨期价差甚至还可以赚钱,因为这个头寸中有额外的看涨期权多头。另一方面,如果 XYZ 停留在相同的价格范围内,那么,这个策略中普通跨期价差的部分就会盈利。即使那手额外的看涨期权在这种情况里有可能会亏损掉一些时间价值,但很容易从其他的 7 手价差的盈利中得到弥补。最糟的情况是 XYZ 价格陡然下跌。不过,在这样的情况里,亏损也不会超过这个价差最初的支出金额。而且,即使是在 XYZ 价格下跌的情况里,也可以采取后续行动。在这个策略中没有裸期权需要交纳保证金,这就使得它对许多较小的投资者具有吸引力。

\section{后续行动}
如果价差交易者决定要在这两种看涨期权比率跨期价差中采取后续行动以保持策略的中性,他只需要看一下这些看涨期权的 delta,保持它们之间的比率中性。这样做有可能意味着他的头寸将从一种类型的跨期价差转换为另一种,从使用裸看涨期权的虚值期权价差转换为使用额外看涨期权多头的实值期权价差,或者是反过来。

虽然从策略的角度来说这样的后续行动并没有错,它维持了中性的比率。但是,从实践上来说,它没有道理,特别是如果价差的初始规模不大的话。如果交易者最初买入 3 手期权,卖出 5 手期权,那他最好仍然使用本前面介绍的后续策略。这个价差没有大到足以需要通过维持 delta 比率中性来进行调整的地步。不过,如果某个大投资者最初买入了 300 手期权并同时卖出了 500 手期权,那么,他在这个价差中就有足够大的盈利能力需要他在一路上做出若干的调整。

与此相似,建立小规模的实值跨期价差的交易者也会发现,即使股票跌到行权价之下,也未必值得对价差进行调整。他知道他的风险是控制在最初支出之内,对一手价差来说,这是一笔小数目。如果 XYZ 下跌,他不需要在头寸中加进裸期权。如果这是一个大交易者所建立的价差的话,那么就应该对其进行调整。这是因为较大规模的头寸让其有更大的调整适应性。