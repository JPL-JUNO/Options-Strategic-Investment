\chapter{对角价差\label{CH:Diagonalizing a Spread}}
对角价差(diagonal spread)是指使用不同行权价和不同到期日的期权来建立的价差。一般而言,价差中多头腿的存续期要比空头腿更长。从保证金的角度来考虑对角价差的定义,多头腿的存续期必须等于或者长于空头腿。

对角头寸有这样的优势:如果价差中卖出的看涨期权无价值到期,交易者可以重新建立这个头寸。如果交易者可以就这个价差出售两次,那他最初在买入较长期的看涨期权所多付的钱就反而变得省钱了。

\section{对角牛市价差}
垂直牛市价差是由买入 1 手行权价较低的看涨期权和卖出 1 手行权价较高的看涨期权组成的,两者的到期日相等。对角牛市价差(diagonal bull spread)同它基本相似,只是交易者\textbf{买入的是 1 手较长期的行权价较低的看涨期权,卖出的是 1 手较短期的行权价较高的看涨期权}。买入和卖出的看涨期权的\textbf{数量仍然相同}。通过将这个价差对角化,如果股票到短期期权到期日时并没有上涨,那么这个头寸在下行方向就有了某种对冲。此外,一旦短期期权到期,这个价差常常可以通过卖出下一个到期日的看涨期权而重新建立起来。

\begin{table}[!ht]
    \centering
    \caption{有下面的价格存在:}
    \begin{tabular}{llllll}
        \hline
            & $K$ & 4 月 & 7 月 & 10 月 & $S_t$ \\ \hline
        XYZ & 30  & 3   & 4   & 5    & 32    \\
        XYZ & 35  & 1   & 1.5 & 2    & 32    \\ \hline
    \end{tabular}
\end{table}

买入行权价为 30 的看涨期权同时卖出行权价为 35 的看涨期权,就可以建立起任何到期日系列上的一手垂直牛市价差。一手对角牛市价差是由买入 7 月 30 或 10 月 30 看涨期权,同时卖出 4 月 35 看涨期权而组成的。

\section{“免费”持有一手看涨期权}
说明这一点的最简单办法是考虑一个对角熊市价差(diagonal bear spread)。
\begin{tcolorbox}
    XYZ 的价格是 32,短期的 4 月 30 看涨期权的售价是 3 点,较长期的 7 月 35 看涨期权的售价是 1.50 点。卖出 4 月 30 看涨期权和买入 7 月 35 看涨期权就可以建立 1 手对角熊市价差。这仍然是 1 手熊市价差,因为卖出的是行权价较低的看涨期权,买入的则是行权价较高的看涨期权。不过,因为买入的看涨期权的存续期比卖出的看涨期权更长,这个价差就是对角的。
\end{tcolorbox}

某个策略家想要使用对角熊市价差,不是因为对角价差在下行方向的细微优势,更可能是因为他有机会能显著降低持有 7 月 35 看涨期权(那个长期的看涨期权)的成本。在这个示例里,这手 7 月 35 看涨期权的成本是 1.50 点,从卖出的 4 月 30 看涨期权中收到的权利金是 3 点。如果该交易者能够从卖出的 4 月 30 看涨期权中盈利 1.50 点,那他就能完全抵消在 7 月期权上的成本。这时他可以坐在一边,并希望标的股票会上涨。如果这样的上涨真的出现了,他就能从多头腿处得到无限的盈利。如果它没有出现,他也没什么损失。
\section{对角后式价差}
在一个相似的策略里,交易者可以就卖出的短期看涨期权而买入多于 1 手的长期看涨期权。

\textbf{看涨期权时间价值的销蚀大部分是在它存续期的最后阶段}。如果这是一手非常长期的期权,销蚀的比率就会相当小。知道这个事实,你就可以明白,交易者有理由想卖出所剩时间不多的期权,因为这样做能给他带来因时减值的最大好处。相应而言,买入较长期的看涨期权就意味着买家在时间价值方面不至于遭受显著的损失,至少在他持有这个期权的前 3 个月里不会如此。对角价差包括了两者:既可以卖出一手短期的看涨期权以得到最大比率的因时减值,又可以买入一手较长期的看涨期权以减小多头腿中的因时减值效果。
\section{看涨期权的总结}
看涨期权被认为是机敏的策略家用来建立各种头寸的一种载体。他可以是看多的,也可以是看空的,可以是激进的,也可以是保守的。此外,他还可以保持中立,从股票在短期内不会有大幅运动这样的可能性中获取盈利。

不熟悉期权的投资者一般来说应当从简单的策略开始,例如,卖出备兑看涨期权,或者直接买入看涨期权。最简单的价差类型是牛市价差、熊市价差和跨期价差。更为精密的投资者可以在他的看涨期权策略中考虑使用比率:就股票做卖出比率,或者是只用期权来做比率价差。

当策略家觉得他懂得了长期看涨期权和短期看涨期权之间、实值看涨期权和虚值看涨期权之间以及看涨期权多头和看涨期权空头之间的风险和回报关系之后,他可以考虑使用最高级的策略类型。这可能包括反向比率价差、对角价差和更为高级的比率类型,例如比率跨期价差。

在前面的章节里有很多信息,有的在细节上的技术性很强。\textbf{投资者应当遵循的模式是,只要没有完全弄懂,就不要使用这个策略}。这并不是说他只懂得策略的盈利性(特别是风险性)等方面就够了。如果要运作高级的策略,交易者还必须能够迅速把握提前指派、大笔股息支付和行权价调整等事件的潜在影响。如果没有完全理解这些事情对其头寸的可能影响,交易者就无法正确地使用这些高级策略。